\documentclass[11pt]{article}
\usepackage[utf8]{inputenc}
\usepackage{booktabs}
\usepackage{hyperref}
\usepackage{aas_macros}

\title{Target of Opportunity proposal for locating a core collapse
  supernova in our galaxy triggered by a neutrino supernova alert}
\author{} \date{May 2018}

\begin{document}

\maketitle

\begin{abstract}

  When a core collapse supernova (CCSN) occurs, over 99\% of its
  gravitational binding energy is released in the form of neutrinos.
  Over a period of 10s of seconds a powerful neutrino flux is emitted
  from the collapsing star.  Due to the high density of material in
  the resulting shock wave, optical photons escaping the expanding
  stellar envelope leave the star and thus arrive at Earth up to
  several hours later than the prompt neutrino signal.  A prompt alert
  will be provided by neutrino experiments, alerting the astronomical
  community that a supernova has occurred in our Galaxy.

  Quickly identifying the location of the supernova on the sky and
  disseminating it to the all available ground and spaced-based
  instruments will be critical to learn as much as possible about the
  event. Using neutrino on atomic-electron scattering, the
  Super-Kamiokande experiment can supply inferred pointing information,
  but only localized to a few degrees for a supernova located near
  the center of our galaxy.  LSST's field of view (FOV) is well
  matched to this initial search box.  LSSTs depth also will allow for
  identifying CCSN even if they fail or are obscured by the dust of
  the galactic plane.

  Such events are only expected to occur a few times per century, and
  studying the time evolution of a supernova from its start is an
  unprecedented opportunity to learn both about the astrophysics of
  these objects and the physics of neutrinos.  This is a proposal
  to, upon receipt of such an alert, use LSST for the rest of that
  night to continuously monitor a pre-identified piece of sky and by
  using difference imaging, identify and announce the location of the
  supernova.
  
\end{abstract}

\section{White Paper Information}

\noindent
Authors: \\

\noindent
Chris Walter - chris.walter@duke.edu \\
Dan Scolnic - dan.scolnic@duke.edu \\
Anze Slozar - anze@bnl.gov \\

\noindent
Categorization: 
\begin{enumerate} 
\item {\bf Science Category:}  Exploring the transient optical sky
\item {\bf Survey Type Category:}  Target of Opportunity observation
\item {\bf Observing Strategy Category:}  A single night, continuous
  observation strategy focused on one few degree field pointing as given by
  a neutrino supernova alert trigger.
\end{enumerate}  

\clearpage

\section{Scientific Motivation}

\begin{footnotesize}
{\it Describe the scientific justification for this white paper in the context
of your field, as well as the importance to the general program of astronomy, 
including the relevance over the next decade. 
Describe other relevant data, and justify why LSST is the best facility for these observations.
(Limit: 2 pages + 1 page for figures.)}
\end{footnotesize}

\vspace{.6in}

As massive ($ > 8 M_\odot$ ) stars approach the end of their lives,
they begin to run out of the hydrogen which has been fueling their
fusion burning.  The resulting loss of pressure results in a
contraction of the star, raising its temperature until the burning of
the helium produced in the previous hydrogen fusion can begin. This
cycle repeats, next burning carbon, neon, oxygen, and silicon until
finally a core of iron remains.  However, iron is at the top of the
nuclear binding energy curve and no energy can be extracted from
fusion.  Without this energy source holding up the material of the
star, it begins to collapse.  The core continues to collapses until it
becomes a neutron star as electrons and protons are forced together
producing neutrons and neutrinos.

These escaping neutrinos form the so called ``neutronization burst''
taking even more energy from the core.  When the core has been
contracted down to nuclear density it is no longer compressible and
bounces back, crashing into the in-falling stellar gas. At this point
the star's ultimate fate depends on its mass.  If massive enough, the
in-falling material crushes the stellar remnant resulting in a black
hole.  Otherwise, the gas bounces back off of the core and a
shock-wave heads out back into the star.  This shock-wave, further
supported by copious neutrinos streaming off the cooling core, drives
out into the star starting a explosion.  Only some hours later will
the shock wave break out of the stellar envelope and become visible as
a supernova explosion.

 The supernova converts the binding energy of the star into energy and
 over 99\% of it is released in the form of neutrinos. First in the
 neutronization burst, and then later in the cooling phase.  Crucially,
 all of these neutrinos escape the star in the first several tens of
 seconds of the explosion. Much can be learned about the dynamics of the
 explosion by studying the neutrino signal.  Even the formation of a
 black hole where a supernova doesn't form should be visible, with a
 cutoff of the neutrino signal.  The internal dynamics of the
 oscillations of the core and the complex neutrino interactions also
 all play a part. The density of neutrinos is so high in the explosion
 that they are thought to even interact with each other resulting in a
 complex phenomenology which is today not completely understood.

 The explosion mechanism itself is also not understood and a large
 fraction of simulations fail to explode at all.  It is still not
 known if this is a feature of the simulations or nature. Studying the
 optical explosion signal in detail from the beginning of the
 explosion will also be key to understanding how the explosion
 proceeds.  No visible CCSN have been seen and measured in the modern
 scientific era.  Although the rate is not completely known CCSN are
 thought to occur roughly two or three times a century.  The
 closest modern CCSN we have observed was SN1987a in the LMC.   In the
 case of SN1987a we also saw the neutrino signal. However, those
 neutrinos were found after the fact in two neutrino detectors: IMB and the
 Kamiokande experiment, not in real time.

Now, a world-wide network of neutrino detectors including the successor
to IMB and Kamiokande, Super-Kamiokande await the neutrinos from a
galactic supernova and have prompt alarms to alert the world if a
supernova neutrino burst has been seen.  Additionally, all of these
experiments are networked together into a system known as SNEWS (the
supernova early warning system) which does a blinded coincidence
between automatic experimental alerts sending out an automated
announcement to the GCN if more than one neutrino experiment has seen
an burst of neutrinos.

Fortunately, the physics of the shock-wave propagation presents us
with an opportunity.  Because if takes so long for the light of the
explosion to escape the stellar envelope, the neutrino signal can
arrive hours before the photons. So, for optical astronomers, the
neutrinos act as an early warning and give the optical community time
to prepare for the arriving light.  This once in a generation
opportunity must not be missed. But, where should we point our
telescopes?  Luckily, the Super-Kamiokande experiment (though one it
it's detection channels) can provide some pointing information.

Most of the neutrino interactions in the water of the Super-Kamiokande
experiment are inverse beta decay (IBD) where a neutrino is captured
by a proton resulting in a positron (which is detected) and a neutron.
The positron in this reaction carries no directional memory of the
incoming neutrino.  However, approximately 1 to 2\% of the neutrino
interactions scatter on atomic electrons and the electrons produced in
this interaction do in fact point back to the supernova.  The fluxes
(and resulting pointings) are somewhat model dependent but studies
have shown that for a CCSN located 10 kpc away it is possible to
determine the direction of the star to within about 3-5
degrees~\cite{2016APh....81...39A}.  Interestingly, systems which form black
holes come from more massive progenitors and result in higher energy
neutrinos that point better (more like 2 degrees).  Near the beginning
of the LSST run Super-Kamiokande is planning to begin running with
Gadolinium dissolved in the water.  Gadolinium has a extremely large
interaction cross-section for neutron interactions and will allow for
efficient tagging of the neutrons in IBD events.  By removing this
non-pointing background the pointing should improve by a factor of a
few (NEED TO GET THIS NUMBER - probably about 1 - 2 degrees).

When the supernova alert comes,  We will know that the optical signal
is coming (PROBABLY PUT GOOD RANGE HERE) and that telescopes should be
waiting and looking.  This is such a rare event that all appropriate
facilities should be ready and trying to identify the light break
out.  Then, when the supernova is identified the information must be
quickly passed to the rest of the community so that the supernova can
be observed in as many wavelengths as possible. (MENTION r-process etc?)

If we are lucky enough that there is a galactic supernova during the
LSST survey and it is visible from the site, LSST is particularly well
suited to do the initial identification and notify the rest of the
community.  There are two reasons for this.  The first is that the FOV
is large. LSST's 3.6 degree FOV is well matched to the initial search
box that would be presented by Super-Kamiokande.  Depending on the
size either a single pointing or a tight pattern of dithering over a
few degrees of the sky is all that will be necessary.  LSST can
continually collect exposures in the region until the explosion is
seen.

Secondly is LSSTs depth. It is true that there are many wild field
surveys that can try to identify the supernova.  If the supernova is
bright, it is indeed the case that other facilities might easily see
the supernova.  But, there are many reasons that a supernova with a
large neutrino flux might not be bright in the optical.  Foremost
among these is that it might be hidden by dust in the galactic plane.
Additionally, based on simulation work, there may be large classes of
supernova that fail to create a explosion~\cite{Evanpaper}. Recent
work has estimated that a supernova located in the disk obsucured by
dust could be as dim as magnitude 25~\cite{2016MNRAS.461.3296N}.
So, we need to account for a huge dynamic range.  Finally it should be
pointed out that LSST already is building a low latency transient
identification and distribution system.

Identifying and studying a galactic supernova would be a scientific
gold-mine for astronomy and particle physics.  The merit of enabling
these studies are very high. The impact on running is minimal.  Over
the course of a 10 year the survey we expect perhaps one night of
disruption.

\section{Technical Description}
\begin{footnotesize}
{\it Describe your survey strategy modifications or proposed observations. Please comment on each observing constraint
below, including the technical motivation behind any constraints. Where relevant, indicate
if the constraint applies to all requested observations or a specific subset. Please note which 
constraints are not relevant or important for your science goals.}
\end{footnotesize}



\subsection{High-level description}
\begin{footnotesize}
{\it Describe or illustrate your ideal sequence of observations.}
\end{footnotesize}

\vspace{.3in}


\subsection{Footprint -- pointings, regions and/or constraints}
\begin{footnotesize}{\it Describe the specific pointings or general region (RA/Dec, Galactic longitude/latitude or 
Ecliptic longitude/latitude) for the observations. Please describe any additional requirements, especially if there
are no specific constraints on the pointings (e.g. stellar density, galactic dust extinction).}
\end{footnotesize}

\subsection{Image quality}
\begin{footnotesize}{\it Constraints on the image quality (seeing).}\end{footnotesize}

Not relevant.

\subsection{Individual image depth and/or sky brightness}
\begin{footnotesize}{\it Constraints on the sky brightness in each image and/or individual image depth for point sources.
Please differentiate between motivation for a desired sky brightness or individual image depth (as 
calculated for point sources). Please provide sky brightness or image depth constraints per filter.}
\end{footnotesize}

The question of image depth is interesting.  What exposure length?
Dynamic adjustment?  


\subsection{Co-added image depth and/or total number of visits}
\begin{footnotesize}{\it  Constraints on the total co-added depth and/or total number of visits.
Please differentiate between motivations for a given co-added depth and total number of visits. 
Please provide desired co-added depth and/or total number of visits per filter, if relevant.}
\end{footnotesize}

\subsection{Number of visits within a night}
\begin{footnotesize}{\it Constraints on the number of exposures (or visits) in a night, especially if considering sequences of visits.  }
\end{footnotesize}

\subsection{Distribution of visits over time}
\begin{footnotesize}{\it Constraints on the timing of visits --- within a night, between nights, between seasons or
between years (which could be relevant for rolling cadence choices in the WideFastDeep. 
Please describe optimum visit timing as well as acceptable limits on visit timing, and options in
case of missed visits (due to weather, etc.). If this timing should include particular sequences
of filters, please describe.}
\end{footnotesize}

\subsection{Filter choice}
\begin{footnotesize}
{\it Please describe any filter constraints not included above.}
\end{footnotesize}

Peak hieght vs NIR for seeing through dust

\subsection{Exposure constraints}
\begin{footnotesize}
{\it Describe any constraints on the minimum or maximum exposure time per visit required (or alternatively, saturation limits).
Please comment on any constraints on the number of exposures in a visit.}
\end{footnotesize}

\subsection{Other constraints}
\begin{footnotesize}
{\it Any other constraints.}
\end{footnotesize}

\subsection{Estimated time requirement}
\begin{footnotesize}
{\it Approximate total time requested for these observations, using the guidelines available at \url{https://github.com/lsst-pst/survey_strategy_wp}.}
\end{footnotesize}

\vspace{.3in}

\begin{table}[ht]
    \centering
    \begin{tabular}{l|l|l|l}
        \toprule
        Properties & Importance \hspace{.3in} \\
        \midrule
        Image quality &     \\
        Sky brightness &  \\
        Individual image depth &   \\
        Co-added image depth &   \\
        Number of exposures in a visit   &   \\
        Number of visits (in a night)  &   \\ 
        Total number of visits &   \\
        Time between visits (in a night) &  \\
        Time between visits (between nights)  &   \\
        Long-term gaps between visits & \\
        Other (please add other constraints as needed) & \\
        \bottomrule
    \end{tabular}
    \caption{{\bf Constraint Rankings:} Summary of the relative importance of various survey strategy constraints. Please rank the importance of each of these considerations, from 1=very important, 2=somewhat important, 3=not important. If a given constraint depends on other parameters in the table, but these other parameters are not important in themselves, please only mark the final constraint as important. For example, individual image depth depends on image quality, sky brightness, and number of exposures in a visit; if your science depends on the individual image depth but not directly on the other parameters, individual image depth would be `1' and the other parameters could be marked as `3', giving us the most flexibility when determining the composition of a visit, for example.}
        \label{tab:obs_constraints}
\end{table}

\subsection{Technical trades}
\begin{footnotesize}
{\it To aid in attempts to combine this proposed survey modification with others, please address the following questions:
\begin{enumerate}
    \item What is the effect of a trade-off between your requested survey footprint (area) and requested co-added depth or number of visits?
    \item If not requesting a specific timing of visits, what is the effect of a trade-off between the uniformity of observations and the frequency of observations in time? e.g. a `rolling cadence' increases the frequency of visits during a short time period at the cost of fewer visits the rest of the time, making the overall sampling less uniform.
    \item What is the effect of a trade-off on the exposure time and number of visits (e.g. increasing the individual image depth but decreasing the overall number of visits)?
    \item What is the effect of a trade-off between uniformity in number of visits and co-added depth? Is there any benefit to real-time exposure time optimization to obtain nearly constant single-visit limiting depth?
    \item Are there any other potential trade-offs to consider when attempting to balance this proposal with others which may have similar but slightly different requests?
\end{enumerate}}
\end{footnotesize}

\section{Performance Evaluation}
\begin{footnotesize}
{\it Please describe how to evaluate the performance of a given survey in achieving your desired
science goals, ideally as a heuristic tied directly to the observing strategy (e.g. number of visits obtained
within a window of time with a specified set of filters) with a clear link to the resulting effect on science.
More complex metrics which more directly evaluate science output (e.g. number of eclipsing binaries successfully
identified as a result of a given survey) are also encouraged, preferably as a secondary metric.
If possible, provide threshold values for these metrics at which point your proposed science would be unsuccessful 
and where it reaches an ideal goal, or explain why this is not possible to quantify. While not necessary, 
if you have already transformed this into a MAF metric, please add a link to the code (or a PR to 
\href{https://github.com/lsst-nonproject/sims_maf_contrib}{sims\_maf\_contrib}) in addition to the text description. (Limit: 2 pages).}
\end{footnotesize}

\vspace{.6in}

\section{Special Data Processing}
\begin{footnotesize}
{\it Describe any data processing requirements beyond the standard LSST Data Management pipelines and how these will be achieved.}
\end{footnotesize}


\section{References}

\bibliographystyle{hunsrt} 
\bibliography{references}

\end{document}
