\documentclass[11pt]{article}

\usepackage{xspace}
\usepackage{amsmath} % has \nobreakdash
\usepackage{graphicx}
\usepackage[utf8]{inputenc}
\usepackage{booktabs}
\usepackage{hyperref}
\usepackage{aas_macros}

\graphicspath{{./figures/}}

\newcommand{\superk}  {Super\nobreakdash-K\xspace}

\title{Target of Opportunity proposal for locating a core collapse
  supernova in our galaxy triggered by a neutrino supernova alert}
\author{D. Scolnic, A. Slosar, C.W. Walter}
\date{ November 2018}

\begin{document}

\maketitle

\begin{abstract}

  When a core collapse supernova (CCSN) occurs, over 99\% of its
  gravitational binding energy is released in the form of neutrinos.
  Over a period of 10s of seconds a powerful neutrino flux is emitted
  from the collapsing star.  When the resulting shock wave finally
  reach the surface of the star, optical photons escaping the
  expanding stellar envelope leave the star and can arrive at Earth up
  to several hours later than the prompt neutrino signal.  A prompt
  alert will be provided by neutrino experiments, alerting the
  astronomical community that a supernova has occurred in our Galaxy.

  Quickly identifying the location of the supernova on the sky and
  disseminating it to the all available ground and spaced-based
  instruments will be critical to learn as much as possible about the
  event. Using neutrino on atomic-electron scattering, the
  Super-Kamiokande experiment can supply inferred pointing information,
  but only localized to a few degrees for a supernova located near
  the center of our galaxy.  LSST's field of view (FOV) is well
  matched to this initial search box.  LSSTs depth also will allow for
  identifying CCSN even if they fail or are obscured by the dust of
  the galactic plane.

  Such events are only expected to occur only a few times per century,
  and studying the time evolution of a supernova from its start is an
  unprecedented opportunity to learn both about the astrophysics of
  these objects and the physics of neutrinos.  This is a proposal to,
  upon receipt of such an alert, use LSST for the rest of that night
  to continuously monitor a pre-identified piece of sky and by using
  difference imaging, identify and announce the location of the
  supernova.
  
\end{abstract}

\section{White Paper Information}

\noindent
Authors: \\

\noindent
Chris Walter - chris.walter@duke.edu \\
Dan Scolnic - dan.scolnic@duke.edu \\
Anze Slozar - anze@bnl.gov \\

\noindent
Categorization: 
\begin{enumerate} 
\item {\bf Science Category:}  Exploring the transient optical sky
\item {\bf Survey Type Category:}  Target of Opportunity observation
\item {\bf Observing Strategy Category:}  A single night, continuous
  observation strategy focused on one few degree field pointing as given by
  a neutrino supernova alert trigger.
\end{enumerate}  

\clearpage

\section{Scientific Motivation}

\begin{footnotesize}
{\it Describe the scientific justification for this white paper in the context
of your field, as well as the importance to the general program of astronomy, 
including the relevance over the next decade. 
Describe other relevant data, and justify why LSST is the best facility for these observations.
(Limit: 2 pages + 1 page for figures.)}
\end{footnotesize}

\vspace{.6in}

As massive ($ > 8 M_\odot$ ) stars approach the end of their lives,
they begin to run out of the hydrogen which has been fueling their
fusion burning.  The resulting loss of pressure results in a
contraction of the star, raising its temperature until the burning of
the helium produced in the previous hydrogen fusion can begin. This
cycle repeats, next burning carbon, neon, oxygen, and silicon until
finally a core of iron remains.  However, iron is at the top of the
nuclear binding energy curve and no energy can be extracted from
fusion.  Without this energy source holding up the material of the
star, it begins to collapse.  The core continues to collapses until it
becomes a neutron star as electrons and protons are forced together
producing neutrons and neutrinos.

These escaping neutrinos form the so called ``neutronization burst''
taking even more energy from the core.  When the core has been
contracted down to nuclear density it is no longer compressible and
bounces back, crashing into the in-falling stellar gas. At this point
the star's ultimate fate depends on its mass.  If massive enough, the
in-falling material crushes the stellar remnant resulting in a black
hole.  Otherwise, the gas bounces back off of the core and a
shock-wave heads out back into the star.  This shock-wave, further
supported by copious neutrinos streaming off the cooling core, drives
out into the star starting a explosion.  Only some hours later will
the shock wave break out of the stellar envelope and become visible as
a supernova explosion.

The supernova converts the binding energy of the star into energy and
over 99\% of it is released in the form of neutrinos. First in the
neutronization burst, and then later in the cooling phase.  Crucially,
all of these neutrinos escape the star in the first several tens of
seconds of the explosion. Much can be learned about the dynamics of
the explosion by studying the neutrino signal.  Even the formation of
a black hole, where a supernova doesn't form, should be visible - with
a cutoff of the neutrino signal.  The internal dynamics of the
oscillations of the core and the complex neutrino interactions also
all play a part. The density of neutrinos is so high in the explosion
that they are thought to even interact with each other resulting in a
complex phenomenology which is today not completely understood.

 The explosion mechanism itself is also not understood and a large
 fraction of simulations fail to explode at all.  It is still not
 known if this is a feature of the simulations or nature. Studying the
 optical explosion signal in detail from the beginning of the
 explosion will also be key to understanding how the explosion
 proceeds.  No visible CCSN have been seen and measured in the modern
 scientific era.  Although the rate is not completely known CCSN are
 thought to occur roughly two or three times a century.  The
 closest modern CCSN we have observed was SN1987a in the LMC.   In the
 case of SN1987a we also saw the neutrino signal. However, those
 neutrinos were found after the fact in two neutrino detectors: IMB and the
 Kamiokande experiment, not in real time.

 Now, a world-wide network of neutrino detectors including the
 successor to IMB and Kamiokande, Super-Kamiokande
 (\superk)~\cite{superk_paper} 1await the neutrinos from a galactic
 supernova and have prompt alarms to alert the world if a supernova
 neutrino burst has been seen.  Additionally, all of these experiments
 are networked together into a system known as SNEWS (the supernova
 early warning system) which does a blinded coincidence between
 automatic experimental alerts sending out an automated announcement
 to the GCN if more than one neutrino experiment has seen an burst of
 neutrinos.

Fortunately, the physics of the shock-wave propagation presents us
with an opportunity.  Because if takes so long for the light of the
explosion to escape the stellar envelope, the neutrino signal can
arrive hours before the photons. So, for optical astronomers, the
neutrinos act as an early warning and give the optical community time
to prepare for the arriving light.  This once in a generation
opportunity must not be missed. But, where should we point our
telescopes?  Luckily, the \superk experiment (though one it
it's detection channels) can provide some pointing information.

Most of the neutrino interactions in the water of the \superk
experiment are inverse beta decay (IBD) where a neutrino is captured
by a proton resulting in a positron (which is detected) and a neutron.
The positron in this reaction carries no directional memory of the
incoming neutrino.  However, approximately 1 to 2\% of the neutrino
interactions scatter on atomic electrons and the electrons produced in
this interaction do in fact point back to the supernova.  The fluxes
(and resulting pointings) are somewhat model dependent but studies
have shown that for a CCSN located 10 kpc away it is possible to
determine the direction of the star to within about 3-5
degrees~\cite{2016APh....81...39A}.  Interestingly, systems which form black
holes come from more massive progenitors and result in higher energy
neutrinos that point better (more like 2 degrees).  Near the beginning
of the LSST run, \superk is planning to begin running with
Gadolinium dissolved in the water.  Gadolinium has a extremely large
interaction cross-section for neutron interactions and will allow for
efficient tagging of the neutrons in IBD events.  By removing this
non-pointing background the pointing should improve by a factor of a
few (NEED TO GET THIS NUMBER - probably about 1 - 2 degrees).

When the supernova alert comes,  We will know that the optical signal
is coming (PROBABLY PUT GOOD RANGE HERE) and that telescopes should be
waiting and looking.  This is such a rare event that all appropriate
facilities should be ready and trying to identify the light break
out.  Then, when the supernova is identified the information must be
quickly passed to the rest of the community so that the supernova can
be observed in as many wavelengths as possible. (MENTION r-process etc?)

If we are lucky enough that there is a galactic supernova during the
LSST survey and it is visible from the site, LSST is particularly well
suited to do the initial identification and notify the rest of the
community.  There are two reasons for this.  The first is that the FOV
is large. LSST's 3.6 degree FOV is well matched to the initial search
box that would be presented by \superk.  Depending on the
size either a single pointing or a tight pattern of dithering over a
few degrees of the sky is all that will be necessary.  LSST can
continually collect exposures in the region until the explosion is
seen.

Secondly is LSSTs depth. It is true that there are many wide field
surveys that can try to identify the supernova.  If the supernova is
bright, it is indeed the case that other facilities might easily see
the supernova.  But, there are many reasons that a supernova with a
large neutrino flux might not be bright in the optical.  Foremost
among these is that it might be hidden by dust in the galactic plane.
Additionally, based on simulation work, there may be large classes of
supernova that fail to create a explosion~\cite{Evanpaper}. Recent
work has estimated that a supernova located in the disk obscured by
dust could be as dim as magnitude 25~\cite{2016MNRAS.461.3296N}.
So, we need to account for a huge dynamic range.  Finally it should be
pointed out that LSST already is building a low latency transient
identification and distribution system.

Identifying and studying a galactic supernova would be a scientific
gold-mine for astronomy and particle physics.  The merit of enabling
these studies are very high. The impact on running is minimal.  Over
the course of a 10 year the survey LSST could expect perhaps one night of
disruption.

\section{Technical Description}
\begin{footnotesize}
{\it Describe your survey strategy modifications or proposed observations. Please comment on each observing constraint
below, including the technical motivation behind any constraints. Where relevant, indicate
if the constraint applies to all requested observations or a specific subset. Please note which 
constraints are not relevant or important for your science goals.}
\end{footnotesize}

\subsection{The neutrino signal}
\label{sec:signal}

As described in the introduction, a CCSN will result in a neutrino
flux which can then be detected.  Each type of detector will see a
different number of events in different channels depending on their
mass and target.  Currently, \superk is the only running experiment
with pointing.  Part way through the LSST 10 year survey, the liquid
argon based DUNE, and the Hyper-Kamiokande experiments will come
online.  The SNEWS system is also now working on adding pointing
information utilizing intra-detector timing.  The travel times across
the Earth from multiple detectors (such as \superk and IceCube) can
triangulate the CCSN position.~\cite{Katepaper}.

Here we primarily focus on the pointing resolution as supplied by the
\superk experiment.  When the neutrinos leave the exploding star they
travel to the Earth and then either pass directly into an underground
neutrino detector or pass through the other side of the Earth to the
detector. The Earth does not attenuate the neutrino signal at all (but
can induce complex neutrino-flavor oscillation effects).  When the
neutrinos reach \superk they interact with water molecules in the
tank.

In the early neutronization burst, the bulk of the emitted neutrinos
are $\nu_e$ but as the cooling phase begins anti-electron neutrinos
and other neutrino flavors are produced in roughly equal numbers as
the energy is lost through the production of neutrino anti-neutrino
pairs from all flavors.  However, due to the differing opacities in
the star for different neutrino species the anti-electron neutrino
decouple from the deepest part of the star and thus have the highest
temperatures and energies.  When those neutrinos reach \superk, the
bulk of the interactions are with protons and proceed through the
inverse beta decay (IBD) reaction:

$$ \overline{\nu}_{e}+ p \rightarrow e^{+} + n. $$

In this reaction the positron is detected through its production of
Cherenkov radiation and the neutral neutron is (typically) not
detected.  Critically, the positron produced in this reaction does not
point back at the supernova.   However, a few percent of the neutrino interactions proceed
through atomic electron scattering:

$$ \nu + e^{-} \rightarrow \nu + e^{-} .$$

This interaction proceeds through both charge-current (via a W boson)
weak interactions where it is sensitive to both electron neutrino and
anti-neutrino fluxes, and neutral-current (via a Z boson) weak
interactions which is sensitive to all neutrino flavors. Unlike the
IBD reactions, these atomic electron scattering interactions {\bf do} point
back to the supernova.  For a more detailed description of the
expected fluxes and neutrino interactions expected in detector refer
to the review~\cite{2012ARNPS..62...81S}.

\subsection{Expected pointing resolution}

The expected pointing resolution in a water Cherenkov detector will
scale with the number of interactions detected.  As explained in
section~\ref{sec:signal} a set of electron scattering interactions which
point back to the supernova will be sitting on top of a background of
non-pointing interactions from the IBD neutrino captures.
Figure~\ref{fig:SK-realtime-monitor-pointing} taken
from~\cite{2016APh....81...39A} shows a typical example set of
interactions from supernova near the galactic center with its
direction reconstructed.  In this figure the electron scattering
interactions are in red, the IBD interactions in blue.

\begin{figure}
  \begin{center}
    \includegraphics[width=3.0in]{SK-realtime-monitor-pointing}
    \caption{}
    \label{fig:SK-realtime-monitor-pointing}
  \end{center}
\end{figure}

For a supernova located 10kpc from the Earth (the galactic center is
approximately 8kpc away) the order of 10,000 neutrino interactions are
expected in \superk.  Supernova that are closer or further away will
have their fluxes scaled simply by scaling to their distance with a
factor of $1/r^2$.  Although 10,000 interactions are typical, expected
fluxes are found to have a range of values by different simulation
groups varying by factors of XXX~\cite{model_references}.  Given a
number of interactions, pointing to the supernova by fitting the
electron scattering signal on top of the IBD background is found to
give a resolution of approximately
%
$$ \Delta \theta = \frac{30^\circ}{\sqrt{N}.}, $$
%
where N is the number of electron-neutrino scattering interactions and
the angular resolution is a half-opening
angle~\cite{2012ARNPS..62...81S}.  With roughly 1\% of 10,000
interactions from a 10kpc supernova being electron scattering events,
this tells us that we should expect a rough pointing resolution of
$3^\circ$.  Closer supernova will have higher numbers of interactions
with better pointing, and those further away will have their
resolution decreased.

A more careful study by the \superk collaboration
in~\cite{2016APh....81...39A} plots a 68\% opening angle coverage as a
function of distance for a few flux
models. Figure~\ref{fig:SK-realtime-pointing-resolution} shows how the
pointing is expected to scale as a function of distance with neutrino
oscillations taken into account for one of the models. At 10kpc the
pointing is near $3^\circ$ as expected.


By the time the LSST survey begins we expect \superk to be doped with
0.02\% GdSo4.  The addition of gadolinium (which has a high neutron
cross-section) to the water will allow the neutron to be tagged in IBD
events thus removing a portion of the non-pointing background and
improving the pointing resolution. [Need more details? phases? Have
private plot from Nakahata...]

\begin{figure}
  \begin{center}
    \includegraphics[width=3.0in]{SK-realtime-pointing-resolution}
    \caption{}
    \label{fig:SK-realtime-pointing-resolution}
  \end{center}
\end{figure}

\subsection{Alert input}


In order to point LSST, the observatory control system (OCS) must first
receive information from the neutrino experiments that a light from a
galactic supernova is about to arrive.  Time is of the essence since
depending on the size and type of the star the breakout time could
range from minutes to several hours~\cite{needcitation}.  There
currently more than one way to receive an alert. If selected LSST must
work with the neutrino community to ensure that the proper information
that LSST needs is being promptly transferred.

There are two broad classes of alerts to consider.  Each experiment
has the option to send its own alert to the astronomical community.
For example, if \superk determines a CCSN in our galaxy has occurred
it might send the following template like text to the Astronomers
Telegram:


\begin{verbatim}
Super-Kamiokande, a 50000 ton water Cherenkov imaging detector
situated 1000 meters underground in the Kamioka mine, Gifu, Japan, has
observed a neutrino burst from a nearby supernova.  Within a fiducial
volume of 22500 tons, preliminary results indicate 5227
neutrino-produced events have been detected with energies greater than
7.0 MeV. An SN1987A-like explosion would be expected to produce such a
signal in Super-Kamiokande if the progenitor star was located at a
distance between 7.55 and 10.36 kpc from Earth. These events were
observed over an interval of 17.9 seconds, with the first event
arriving at 2017 Nov 2.318437 UT. The estimated supernova direction is
R.A. = 110 (degrees) and Dec.= 6 (degrees), within 3.29, 4.72 and 5.62
degrees for, respectively, 68, 90 and 95% C.L. error circles. 
The probability to have the SN located within 2, 5, and 10 degrees 
of the central position is 0.36, 0.92 and 1.00, respectively.
\end{verbatim}

C.W. Walter is a member of both \superk and the LSST project and
the \superk spokesperson has expressed interest in personal
conversation to supply direct information to the LSST OCS in whatever
form is most appropriate.

For many years the neutrino community as a whole has being preparing
for a galactic supernova through the creation of the Supernova Early
Warning System (SNEWS)~\cite{2004NJPh....6..114A}\footnote{Website and
  more information at \url{https://snews.bnl.gov}}. SNEWS acts as a
broker and a blinded system to look for coincidences in time between
supernova alarms coming from different neutrino experiments. If one is
seen, then they alert the entire astronomical community through several
channels. This reduces the false coincidence rate to less than one
alert per century. SNEWS can also act as a broker passing alerts from
individual experiments to their alert system

SNEWS has several ways of making announcements to the community.  They
also give a direct connection to the IceCube experiment which benefits
from an external trigger.  A direct connection to LSST could also be
arranged.  Current  alerts include announcements to the GCN (a
template is seen below):

% SNEWS template
\begin{verbatim}
TITLE:         GCN/SNEWS EVENT NOTICE
NOTICE_DATE:   Tue 26 Jun 18 16:00:08 UT
NOTICE_TYPE:   TEST COINCIDENCE
TRIGGER_NUM:   1000182
EVENT_RA:      Undefined (J2000),
              Undefined (current),
              Undefined (1950)
EVENT_DEC:     Undefined (J2000),
              Undefined (current),
              Undefined (1950)
EVENT_ERROR:   360.0 [deg radius, statistical plus systematic], 68.00% containment
EVENT_FLUENCE: 0 [neutrinos]
EVENT_TIME:    57601.00 SOD {16:00:01.00} UT
EVENT_DATE:    18295 TJD;   177 DOY;   18/06/26
EVENT_DUR:     0.00 [sec]
EXPT:          Detector_A Good, Detector_B Good, Detector_D Possible, Detector_E Good, Detector_F Possible, 
SUN_POSTN:      95.45d {+06h 21m 49s}  +23.34d {+23d 20' 30"}
SUN_DIST:      Undefined [deg]
MOON_POSTN:    257.26d {+17h 09m 02s}  -19.06d {-19d 03' 38"}
MOON_DIST:     Undefined [deg]
MOON_ILLUM:    98 [%]
GAL_COORDS:    Undefined,Undefined [deg] galactic lon,lat of the event
ECL_COORDS:    Undefined,Undefined [deg] ecliptic lon,lat of the event
COMMENTS:      SNEWS Event without RA,Dec coordinates.  
COMMENTS:      This is a Test COINCIDENCE notice.  It is NOT a Real event.  
COMMENTS:      This is a Test COINCIDENCE notice.  The EXPT labels have been anonymized.  
COMMENTS:         
COMMENTS:      RA,Dec fields undefined.  
COMMENTS:      For more information see:  
COMMENTS:
\end{verbatim}

% SNEWS direct template

\noindent
and email alerts such as the template below.

\begin{verbatim}
-----BEGIN PGP SIGNED MESSAGE-----
Hash: SHA1

- ---------------------------------------------
*** SNEWS ALERT ***
Coincidence rating: GOLD
Alarms in the coincidence:
Experiment: 5 LVD
Level: GOOD
Time: Jan 02 2006 22:34:37.000000000
Duration:   10.00
No. of signal events:    0.00
Right Ascension:    0.00
Declination:    0.00
Error:  360.10
- ---------------------------------------------
Experiment: 3 SNO
Level: GOOD 
Time: Jan 02 2006 22:34:37.000000000
Duration:   10.00
No. of signal events:    0.00
Right Ascension:    0.00
Declination:    0.00
Error:  360.10
- ---------------------------------------------
Experiment: 1 Super-K
Level: POSSIBLE 
Time: Jan 02 2006 22:34:37.000000000
Duration:   10.00
No. of signal events:    0.00
Right Ascension:    13.00
Declination:    -3.00
Error:  4.0
- ---------------------------------------------

For information, see web page http://snews.bnl.gov/
-----BEGIN PGP SIGNATURE-----
Version: GnuPG v1.4.9 (GNU/Linux)

iD8DBQFMhguY4A2qNGjfk/cRAp+DAKD2cFdN4aHZomU87XhhA2r7GalWcACgt/oM
ffObwWjd44FA6kx5gx/RLDQ=
=DtVE
-----END PGP SIGNATURE-----
\end{verbatim}

Currently, a fast alarm goes directly from \superk to the SNEWS system
with no human intervention.  However, that alarm does not contain
pointing information.  Now, that information is released only after a
virtual meeting of \superk collaborators to confirm the alarm.
However, it is recognized that this step slows down dissemination, so
discussion is starting on the best way to pass this information either
directly to projects like LSST or through systems like SNEWS.

LSST could decide to require multiple or single experimental alarms
still taking the pointing system from \superk.  In the future, SNEWS
is also expected to supply pointing based on triangulation using
timing and this information could be combined with the electron
scattering signal.

\subsection{High-level description}
\begin{footnotesize}
{\it Describe or illustrate your ideal sequence of observations.}
\end{footnotesize}

\vspace{.3in}


Describe sequence here:

\begin{itemize}
\item Receive alert
\item Go into SN watch mode
\item choose exposure, filter and dither pattern
\item Perform fast DIA analysis to locate CCSN candidate
\item Pass information about the CCSN candidate to community o allow
  other followup observations.
\item Follow for some time for the rest of the night of observation.
\end{itemize}

\subsection{Footprint -- pointings, regions and/or constraints}
\begin{footnotesize}{\it Describe the specific pointings or general region (RA/Dec, Galactic longitude/latitude or 
Ecliptic longitude/latitude) for the observations. Please describe any additional requirements, especially if there
are no specific constraints on the pointings (e.g. stellar density, galactic dust extinction).}
\end{footnotesize}

Even with a large neutrino signal the optical signal from the CCSN
could be quite dim.  This is a place where LSST will make a
particularly vital contribution.  The explosion could fail or form a
black hole~\cite{evanpaper?} or the supernova could be hidden in the
dust of the galactic plane.  The expected range of brightnesses are
explored in~\cite{2016MNRAS.461.3296N} and
figure~\ref{fig:multimessenger-comparison} taken from that paper shows
the reach of LSST for the dimmest supernova compared to other facilities.

\begin{figure}
  \begin{center}
    \includegraphics[width=2.0in]{multimessenger-comparison}
    \caption{Caption}
    \label{fig:multimessenger-comparison}
  \end{center}
\end{figure}

Depending on the pointing resolution the OCS would either point the
telescope at one point in the sky taking a continuous set of
exposures, or likely put it in a tight dither pattern covering a few
FOVs worth of area.

\subsection{Image quality}
\begin{footnotesize}{\it Constraints on the image quality (seeing).}\end{footnotesize}

The image quality is not relevant.

\subsection{Individual image depth and/or sky brightness}
\begin{footnotesize}{\it Constraints on the sky brightness in each image and/or individual image depth for point sources.
Please differentiate between motivation for a desired sky brightness or individual image depth (as 
calculated for point sources). Please provide sky brightness or image depth constraints per filter.}
\end{footnotesize}

Not relevant.

\subsection{Co-added image depth and/or total number of visits}
\begin{footnotesize}{\it  Constraints on the total co-added depth and/or total number of visits.
Please differentiate between motivations for a given co-added depth and total number of visits. 
Please provide desired co-added depth and/or total number of visits per filter, if relevant.}
\end{footnotesize}

Not relevant (?).  Should we try to build a co-add?

\subsection{Number of visits within a night}
\begin{footnotesize}{\it Constraints on the number of exposures (or visits) in a night, especially if considering sequences of visits.  }
\end{footnotesize}

This mode would be envisioned to completely take over the facility for
one dark period after the alarm arrives.

\subsection{Distribution of visits over time}
\begin{footnotesize}{\it Constraints on the timing of visits --- within a night, between nights, between seasons or
between years (which could be relevant for rolling cadence choices in the WideFastDeep. 
Please describe optimum visit timing as well as acceptable limits on visit timing, and options in
case of missed visits (due to weather, etc.). If this timing should include particular sequences
of filters, please describe.}
\end{footnotesize}

Note relevant.

\subsection{Filter choice}
\begin{footnotesize}
{\it Please describe any filter constraints not included above.}
\end{footnotesize}

Peak height vs NIR for seeing through dust. Need to look at light
curves and throughput.

\subsection{Exposure constraints}
\begin{footnotesize}
{\it Describe any constraints on the minimum or maximum exposure time per visit required (or alternatively, saturation limits).
Please comment on any constraints on the number of exposures in a visit.}
\end{footnotesize}

Dynamic exposure? What happens if it is too bright?


\subsection{Other constraints}
\begin{footnotesize}
{\it Any other constraints.}
\end{footnotesize}

\subsection{Estimated time requirement}
\begin{footnotesize}
{\it Approximate total time requested for these observations, using the guidelines available at \url{https://github.com/lsst-pst/survey_strategy_wp}.}
\end{footnotesize}

We expect no more than one night in the 10 year survey would be
affected by this program. With an expected rate of 3 per century there
is a 22\% chance one CCSN would be seen and a 3\% chance there could
be two.

\vspace{.3in}

\begin{table}[ht]
    \centering
    \begin{tabular}{l|l|l|l}
        \toprule
        Properties & Importance \hspace{.3in} \\
        \midrule
        Image quality &     \\
        Sky brightness &  \\
        Individual image depth &   \\
        Co-added image depth &   \\
        Number of exposures in a visit   &   \\
        Number of visits (in a night)  &   \\ 
        Total number of visits &   \\
        Time between visits (in a night) &  \\
        Time between visits (between nights)  &   \\
        Long-term gaps between visits & \\
        Other (please add other constraints as needed) & \\
        \bottomrule
    \end{tabular}
    \caption{{\bf Constraint Rankings:} Summary of the relative importance of various survey strategy constraints. Please rank the importance of each of these considerations, from 1=very important, 2=somewhat important, 3=not important. If a given constraint depends on other parameters in the table, but these other parameters are not important in themselves, please only mark the final constraint as important. For example, individual image depth depends on image quality, sky brightness, and number of exposures in a visit; if your science depends on the individual image depth but not directly on the other parameters, individual image depth would be `1' and the other parameters could be marked as `3', giving us the most flexibility when determining the composition of a visit, for example.}
        \label{tab:obs_constraints}
\end{table}

\subsection{Technical trades}
\begin{footnotesize}
{\it To aid in attempts to combine this proposed survey modification with others, please address the following questions:
\begin{enumerate}
    \item What is the effect of a trade-off between your requested survey footprint (area) and requested co-added depth or number of visits?
    \item If not requesting a specific timing of visits, what is the effect of a trade-off between the uniformity of observations and the frequency of observations in time? e.g. a `rolling cadence' increases the frequency of visits during a short time period at the cost of fewer visits the rest of the time, making the overall sampling less uniform.
    \item What is the effect of a trade-off on the exposure time and number of visits (e.g. increasing the individual image depth but decreasing the overall number of visits)?
    \item What is the effect of a trade-off between uniformity in number of visits and co-added depth? Is there any benefit to real-time exposure time optimization to obtain nearly constant single-visit limiting depth?
    \item Are there any other potential trade-offs to consider when attempting to balance this proposal with others which may have similar but slightly different requests?
\end{enumerate}}
\end{footnotesize}

Not relevant.

\section{Performance Evaluation}
\begin{footnotesize}
{\it Please describe how to evaluate the performance of a given survey in achieving your desired
science goals, ideally as a heuristic tied directly to the observing strategy (e.g. number of visits obtained
within a window of time with a specified set of filters) with a clear link to the resulting effect on science.
More complex metrics which more directly evaluate science output (e.g. number of eclipsing binaries successfully
identified as a result of a given survey) are also encouraged, preferably as a secondary metric.
If possible, provide threshold values for these metrics at which point your proposed science would be unsuccessful 
and where it reaches an ideal goal, or explain why this is not possible to quantify. While not necessary, 
if you have already transformed this into a MAF metric, please add a link to the code (or a PR to 
\href{https://github.com/lsst-nonproject/sims_maf_contrib}{sims\_maf\_contrib}) in addition to the text description. (Limit: 2 pages).}
\end{footnotesize}

Not relevant.  Performance would be measured in successfully
identifying and quickly notifying the community of the location of the
optical supernova counterpart.


\vspace{.6in}

\section{Special Data Processing}
\begin{footnotesize}
{\it Describe any data processing requirements beyond the standard LSST Data Management pipelines and how these will be achieved.}
\end{footnotesize}

A version of the DIA pipeline would need to be utilized to make the
initial identification.

\section{References}

\bibliographystyle{hunsrt} 
\bibliography{references}

\end{document}
