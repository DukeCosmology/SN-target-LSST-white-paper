\documentclass[12pt, letterpaper]{article}

\usepackage{xspace}
\usepackage{amsmath} % has \nobreakdash
\usepackage{graphicx}
\usepackage[utf8]{inputenc}
\usepackage{booktabs}
\usepackage{hyperref}
\usepackage{aas_macros}

\usepackage[top=1in, bottom=1.5in, left=1in, right=1in]{geometry}

\graphicspath{{./figures/}}

\newcommand{\superk}  {Super\nobreakdash-K\xspace}

\title{LSST Target of Opportunity proposal for locating a core collapse
  supernova in our galaxy triggered by a neutrino supernova alert}
\author{ C.W. Walter, D. Scolnic, A. Slosar}
\date{ November 2018}

\begin{document}

\maketitle

\begin{abstract}

  A few times a century, a core collapse supernova (CCSN) occurs in
  our galaxy. When such galactic CCSN happen, over 99\% of its
  gravitational binding energy is released in the form of neutrinos.
  Over a period of 10s of seconds a powerful neutrino flux is emitted
  from the collapsing star.  When the exploding shock wave finally
  reaches the surface of the star, optical photons escaping the
  expanding stellar envelope leave the star and eventually arrive at
  Earth as a visible brightening.

  By combining the multi-messenger signal from optical, neutrino, and
  gravitational waves, we afford an unprecedented opportunity to learn
  about the astrophysics of these rare objects. Carefully measuring
  the optical light curve of the explosion will give critical
  information about the size and composition of the progenitor star
  and help understand the dynamics of the explosion.

  Crucially, although the neutrino signal is prompt, the time to the
  shock wave breakout can be minutes to many hours later.  This means
  that the neutrino signal will serve as an alert, warning the
  optical astronomy community the light from the explosion is coming.  
  Quickly identifying the location of the supernova on the sky and
  disseminating it to the all available ground and spaced-based
  instruments will be critical to learn as much as possible about the
  event.

  Some neutrino experiments can report pointing information for these
  galactic CCSN. In particular, the Super-Kamiokande experiment can
  point to a few degrees for CCSN near the center of our galaxy.  A
  CCCN located 10~kpc from Earth is expected to result in a pointing
  resolution of the order of $3^\circ$.  LSST's field of view (FOV) is
  well matched to this initial search box.  LSSTs depth is also
  uniquely suited for identifying CCSN even if they fail or are
  obscured by the dust of the galactic plane.

  This is a proposal to, upon receipt of such an alert, prioritize the
  use of LSST for a full day of observing to continuously monitor a
  pre-identified region of sky and, by using difference imaging,
  identify and announce the location of the supernova. In this
  proposal, we propose to use one night (approximately .03\% of the
  survey period) if a galactic supernova occurs.  Based on estimates
  of the rate of such CCSN there is approximately a 20\% chance that a
  CCSN will occur during the survey period.
  
\end{abstract}

\section{White Paper Information}

\noindent
Authors: \\

\noindent
Chris Walter - chris.walter@duke.edu \\
Dan Scolnic - dan.scolnic@duke.edu \\
Anze Slosar - anze@bnl.gov \\

\noindent
Categorization: 
\begin{enumerate} 
\item {\bf Science Category:}  Exploring the transient optical sky
\item {\bf Survey Type Category:}  Target of Opportunity observation
\item {\bf Observing Strategy Category:}  A single night, continuous
  observation strategy focused on one few degree field pointing as given by
  a neutrino supernova alert trigger.
\end{enumerate}  

\clearpage

\section{Scientific Motivation}
\label{sec:motivation}

No visible galactic CCSN have been seen and measured in the modern
scientific era. It is thought that CCSN occur roughly two or three
times a century~\cite{1994ApJS...92..487T, 2001ASSL..264..199C}.  The
closest modern CCSN we have observed was SN1987a in the LMC.  Using a
alarm from neutrino detectors as a trigger, LSST can quickly identify
and characterize a galactic core-collapse supernova and then notify
the rest of the astronomical community which can then track it in
multiple wavelengths from the ground and space.  The early
identification of the supernova optical counterpart will be key in an
extensive program of multi-messenger astronomy.

How a neutrino-based supernova alarm can be used to prepare for the
arrival of an optical signal in LSST can be understood from the basic
sequence of CCSN formation.  As massive ($ > 8 M_\odot$ ) stars
approach the end of their lives, they begin to run out of the hydrogen
which has been fueling their fusion burning.  The resulting loss of
pressure results in a contraction of the star, raising its temperature
until the burning of the helium produced in the previous hydrogen
fusion can begin. This cycle repeats, next burning carbon, neon,
oxygen, and silicon until finally a core of iron remains. At this
point, the star's ultimate fate depends on its mass.  If massive
enough, the in-falling material crushes the stellar remnant resulting
in a black hole.  Otherwise, the gas bounces back off of the
compressing core and a shock-wave begins moving towards the surface of
the star.  This shock-wave, further supported by copious neutrinos
streaming off the cooling core, drives out into the star starting a
explosion.  Depending on the size of the star, only minutes to many
hours later will the shock wave actually break out of the stellar
envelope and become visible as a supernova explosion to optical
telescopes.

Studying a galactic supernova in detail is an amazing opportunity to
do multi-messenger astronomy~\cite{2016MNRAS.461.3296N}.  Neutrino,
gravitational wave, and optical signals all tell us something unique
about the system.  The supernova converts the binding energy of the
star into energy and over 99\% of it is released in the form of
neutrinos.  Crucially, all of these neutrinos escape the star in the
first several tens of seconds of the explosion. Much can be learned
about the dynamics of the explosion by studying the neutrino signal.
Even the formation of a black hole, where the neutrino signal will be
abruptly terminated, should be visible~\cite{2011ApJ...730...70O,
  2017hsn..book.1555O}.

Additionally, studying the optical explosion signal in detail from the
beginning of the explosion will probe how the explosion proceeds and
will give crucial insights into the character, composition and size of
the progenitor star.  This will allow us to better understand the
final stages of stellar evolution and the environment that exists as
the collapse begins.  Examples of strategies to constrain the
progenitor characteristics and explosion dynamics from light curves
include~\cite{2010ApJ...725..904N, 2017NatPh..13..510Y,
  2018ApJ...856..146A}.  Some models of black hole formation even
include a much reduced, but possibly visible, electromagnetic
signal~\cite{2013ApJ...769..109L}. With the progenitor information
gained from the light curve we can connect what is happening on the
outside of the star to what is happening on the inside as measured by
the neutrino and gravity wave signal.
 
A world-wide network of neutrino detectors including Super-Kamiokande
(\superk)~\cite{2003NIMPA.501..418F} have prompt alarms to alert the
world if a supernova neutrino burst has been seen.  Additionally, all
of these experiments are networked together into a system known as
SNEWS (the supernova early warning system; website and more
information at \url{https://snews.bnl.gov})~\cite{2004NJPh....6..114A}
which does a blinded coincidence between automatic experimental alerts
sending out an automated announcement to the GCN if more than one
neutrino experiment has seen a burst of neutrinos.

Some detectors only can report the time of a neutrino burst, and do
not have the ability to supply directional information.  In the near
future, there will be new detectors with pointing ability and the
SNEWS system will also add pointing information utilizing
intra-detector timing by using the fact that travel times across the
Earth from multiple detectors (such as \superk and IceCube) can
triangulate the CCSN position~\cite{2018JCAP...04..025B} and [N. Linzer
and K. Scholberg in preparation].  However, currently,
\superk is the only running experiment with pointing ability, and we
use its performance for this proposal. Most of the neutrino
interactions in the water of the \superk experiment are inverse beta
decay (IBD) ($ \overline{\nu}_{e}+ p \rightarrow e^{+} + n $) where a
neutrino is captured by a proton resulting in a positron (which is
detected through its Cherenkov radiation) and a neutron.  The positron
in this reaction carries no directional memory of the incoming
neutrino. However, a few percent of the neutrino interactions proceed
through atomic electron scattering:
$\nu + e^{-} \rightarrow \nu + e^{-} .$

Unlike the IBD reactions, these atomic electron scattering
interactions {\bf do} point back to the supernova.  For a more
detailed description of the expected fluxes from each neutrino type
and neutrino interactions expected in detector refer to the
review~\cite{2012ARNPS..62...81S}.
Figure~\ref{fig:SK-realtime-monitor-pointing} taken
from~\cite{2016APh....81...39A} shows a typical example set of
interactions from supernova near the galactic center with its
direction reconstructed.  In this figure, the electron scattering
interactions are in red, the IBD interactions in blue.  The fluxes
(and resulting pointings) are somewhat model dependent but studies
have shown that for a CCSN located 10 kpc away it is possible to
determine the direction of the star to within about 3-5
degrees~\cite{2016APh....81...39A}.  Closer or more luminous CCSN will
have better pointing.  The expected time delay ranges from minutes to
a day depending on the mass of the progenitor
star. Figure~\ref{fig:delay-times} taken
from~\cite{2013ApJ...778...81K} shows the range of expected times.
SN1987a progenitor was thought to be a blue giant with a time delay of
around 3 hours~\cite{ISAWTHISSOMWHERE}.

LSST is particularly well suited to do the initial CCSN
identification.  LSST's large 3.6 degree FOV is well matched to the
initial search region that would be presented by \superk.  Depending
on the size, either a single pointing or a tight pattern of dithering
over a few degrees of the sky is all that will be necessary.  LSST can
continually collect exposures in the region until the explosion is
seen. Next is LSSTs depth. It is true that there are many wide field
surveys that can try to identify the supernova.  If the supernova is
bright, it is indeed the case that other facilities might easily see
the supernova.  But, even with a large neutrino signal, the optical
signal from the CCSN could be quite dim.  Recent work has estimated
that a supernova located in the disk obscured by dust could be as dim
as magnitude 25~\cite{2016MNRAS.461.3296N}.  This is a place where
LSST will make a particularly vital contribution.  The explosion could
fail or form a black hole~\cite{2011ApJ...730...70O,
  2017hsn..book.1555O} or the supernova could be hidden in the dust of
the galactic plane.  The expected range of brightnesses are explored
in~\cite{2016MNRAS.461.3296N},
figure~\ref{fig:multimessenger-comparison} taken from that paper shows
the reach of LSST for the dimmest supernova compared to other
facilities.

Finally, to quickly identify the CCSN, in the case that we do not
acquire images after the alert but before the optical light arrives,
deep image templates of the area will be necessary for image
subtraction and candidate identification.  Likely the depth of these
templates will be the main limiting factor for identifying faint
candidates.  LSST should have templates across the sky after Y1.  The
depth $\times$ area on average across the sky will be deeper for LSST
than any other survey.

Identifying and studying a galactic supernova would be a scientific
gold-mine for astronomy and particle physics.  The merit of enabling
these studies are very high. The impact on running is minimal.  Using
the rate of three per century there is a 20\% chance that we would
receive a neutrino alarm. In that case we advocate for a strategy of a
full day of observing with follow-up over next few days to ensure the
candidate.  Depending on how bright it is, we would quickly hand off
to other ground and space-based telescopes.  It would take
approximately 0.3\% of the survey's time to contribute to a major
discovery.

\clearpage

\begin{figure}
  \begin{center}
    \includegraphics[width=4.0in]{SK-realtime-monitor-pointing}
    \caption{Caption... replace with real figure and no screen grab.}
    \label{fig:SK-realtime-monitor-pointing}
  \end{center}
\end{figure}

\begin{figure}
  \begin{center}
    \begin{minipage}[b]{3.1in}
      \includegraphics[width=3.0in]{apj487119f2_hr}
      \caption{Caption}
      \label{fig:delay-times}
    \end{minipage}
    \begin{minipage}[b]{3.1in}
      \includegraphics[width=3.0in]{fig9}
      \caption{Caption}
      \label{fig:multimessenger-comparison}
    \end{minipage}
  \end{center}
\end{figure}

\clearpage

\newpage
\section{Technical Description}

In order to describe the survey strategy for footprint, tiling method,
depth, and observation frequency it is important to understand the
expected time delay and pointing signal and the form of the alarm
signal.  Those are first summarized here.

\subsection{Expected Time Delay}

The time delay between the neutrino alarm and the light signal
reaching LSST can range from minutes in the case of Wolf-Rayet stars,
to hours for blue supergiants, all the way up to a day or more for red
supergiants.  The time is set by the radius of the star when the
collapse happens, as that sets the distance that the shock wave must
travel.  Figure~\ref{fig:delay-times} taken
from~\cite{2013ApJ...778...81K} shows how the breakout time of the
shockwave varies for different classes of stellar objects. Table~2
in~\cite{2015ApJ...814...63M}  calculates the delay times for various
modeled red supergiants.

\subsection{Expected pointing resolution}

The expected pointing resolution in a water Cherenkov detector will
scale with the number of interactions detected.  As explained in
section~\ref{sec:motivation} a set of electron scattering interactions which
point back to the supernova will be sitting on top of a background of
non-pointing interactions from the IBD neutrino captures.

For a supernova located 10kpc from the Earth (the galactic center is
approximately 8kpc away) the order of 10,000 neutrino interactions are
expected in \superk.  Supernova that are closer or further away will
have their fluxes scaled simply by scaling to their distance with a
factor of $1/r^2$.  Although 10,000 interactions are typical, expected
fluxes are found to have a range of values by different simulation
groups varying by factors of XXX~\cite{model_references}.  Given a
number of interactions, pointing to the supernova by fitting the
electron scattering signal on top of the IBD background is found to
give a resolution of approximately
%
$$ \Delta \theta = \frac{30^\circ}{\sqrt{N}.}, $$
%
where N is the number of electron-neutrino scattering interactions and
the angular resolution is a half-opening
angle~\cite{2012ARNPS..62...81S}.  With roughly 1\% of 10,000
interactions from a 10kpc supernova being electron scattering events,
this tells us that we should expect a rough pointing resolution of
$3^\circ$.  Closer supernova will have higher numbers of interactions
with better pointing, and those further away will have their
resolution decreased.

A more careful study by the \superk collaboration
in~\cite{2016APh....81...39A} plots a 68\% opening angle coverage as a
function of distance for a few flux
models. Figure~\ref{fig:SK-realtime-pointing-resolution} shows how the
pointing is expected to scale as a function of distance with neutrino
oscillations taken into account for one of the models. At 10kpc the
pointing is near $3^\circ$ as expected.

By the time the LSST survey begins we expect \superk to be doped with
0.02\% GdSo4.  The addition of gadolinium (which has a high neutron
cross-section) to the water will allow the neutron to be tagged in IBD
events thus removing a portion of the non-pointing background and
improving the pointing resolution. [Need more details? phases? Have
private plot from Nakahata...]

\begin{figure}
  \begin{center}
    \includegraphics[width=3.0in]{SK-realtime-pointing-resolution}
    \caption{}
    \label{fig:SK-realtime-pointing-resolution}
  \end{center}
\end{figure}

\subsection{Alert input}

In order to point LSST, the observatory control system (OCS) must
first receive information from the neutrino experiments that a light
from a galactic supernova is about to arrive.  Time is of the essence
since depending on the size and type of the star the breakout time
could range from minutes to many hours~\cite{2013ApJ...778...81K}.
There is currently more than one way to receive an alert. If selected
LSST must work with the neutrino community to ensure that the
information that LSST needs is being promptly transferred.

There are two broad classes of alerts to consider.  Each experiment
has the option to send its own alert to the astronomical community.
For example, if \superk determines a CCSN in our galaxy has occurred
it might send the following template like text to the Astronomers
Telegram:

\begin{verbatim}
Super-Kamiokande, a 50000 ton water Cherenkov imaging detector
situated 1000 meters underground in the Kamioka mine, Gifu, Japan, has
observed a neutrino burst from a nearby supernova.  Within a fiducial
volume of 22500 tons, preliminary results indicate 5227
neutrino-produced events have been detected with energies greater than
7.0 MeV. An SN1987A-like explosion would be expected to produce such a
signal in Super-Kamiokande if the progenitor star was located at a
distance between 7.55 and 10.36 kpc from Earth. These events were
observed over an interval of 17.9 seconds, with the first event
arriving at 2017 Nov 2.318437 UT. The estimated supernova direction is
R.A. = 110 (degrees) and Dec.= 6 (degrees), within 3.29, 4.72 and 5.62
degrees for, respectively, 68, 90 and 95% C.L. error circles. 
The probability to have the SN located within 2, 5, and 10 degrees 
of the central position is 0.36, 0.92 and 1.00, respectively.
\end{verbatim}

C.W. Walter is a member of both \superk and the LSST project and
\superk has expressed interest (in personal communications) to CWW in
supplying direct information to the LSST OCS in whatever form is most
appropriate.

For many years the neutrino community as a whole has being preparing
for a galactic supernova through the creation of the Supernova Early
Warning System (SNEWS).  SNEWS acts as a
broker and a blinded system to look for coincidences in time between
supernova alarms coming from different neutrino experiments. If one is
seen, then they alert the entire astronomical community through several
channels. This reduces the false coincidence rate to less than one
alert per century. SNEWS can also act as a broker passing alerts from
individual experiments to their alert system

SNEWS has several ways of making announcements to the community.  They
also give a direct connection to the IceCube experiment which benefits
from an external trigger.  A direct connection to LSST could also be
arranged.  Current  alerts include announcements to the GCN (a
template is seen below):

% SNEWS template
\begin{verbatim}
TITLE:         GCN/SNEWS EVENT NOTICE
NOTICE_DATE:   Tue 26 Jun 18 16:00:08 UT
NOTICE_TYPE:   TEST COINCIDENCE
TRIGGER_NUM:   1000182
EVENT_RA:      Undefined (J2000),
              Undefined (current),
              Undefined (1950)
EVENT_DEC:     Undefined (J2000),
              Undefined (current),
              Undefined (1950)
EVENT_ERROR:   360.0 [deg radius, statistical plus systematic], 68.00% containment
EVENT_FLUENCE: 0 [neutrinos]
EVENT_TIME:    57601.00 SOD {16:00:01.00} UT
EVENT_DATE:    18295 TJD;   177 DOY;   18/06/26
EVENT_DUR:     0.00 [sec]
EXPT:          Detector_A Good, Detector_B Good, Detector_D Possible, Detector_E Good, Detector_F Possible, 
SUN_POSTN:      95.45d {+06h 21m 49s}  +23.34d {+23d 20' 30"}
SUN_DIST:      Undefined [deg]
MOON_POSTN:    257.26d {+17h 09m 02s}  -19.06d {-19d 03' 38"}
MOON_DIST:     Undefined [deg]
MOON_ILLUM:    98 [%]
GAL_COORDS:    Undefined,Undefined [deg] galactic lon,lat of the event
ECL_COORDS:    Undefined,Undefined [deg] ecliptic lon,lat of the event
COMMENTS:      SNEWS Event without RA,Dec coordinates.  
COMMENTS:      This is a Test COINCIDENCE notice.  It is NOT a Real event.  
COMMENTS:      This is a Test COINCIDENCE notice.  The EXPT labels have been anonymized.  
COMMENTS:         
COMMENTS:      RA,Dec fields undefined.  
COMMENTS:      For more information see:  
COMMENTS:
\end{verbatim}

% SNEWS direct template

\noindent
and email alerts such as the template below.

\begin{verbatim}
-----BEGIN PGP SIGNED MESSAGE-----
Hash: SHA1

- ---------------------------------------------
*** SNEWS ALERT ***
Coincidence rating: GOLD
Alarms in the coincidence:
Experiment: 5 LVD
Level: GOOD
Time: Jan 02 2006 22:34:37.000000000
Duration:   10.00
No. of signal events:    0.00
Right Ascension:    0.00
Declination:    0.00
Error:  360.10
- ---------------------------------------------
Experiment: 3 SNO
Level: GOOD 
Time: Jan 02 2006 22:34:37.000000000
Duration:   10.00
No. of signal events:    0.00
Right Ascension:    0.00
Declination:    0.00
Error:  360.10
- ---------------------------------------------
Experiment: 1 Super-K
Level: POSSIBLE 
Time: Jan 02 2006 22:34:37.000000000
Duration:   10.00
No. of signal events:    0.00
Right Ascension:    13.00
Declination:    -3.00
Error:  4.0
- ---------------------------------------------

For information, see web page http://snews.bnl.gov/
-----BEGIN PGP SIGNATURE-----
Version: GnuPG v1.4.9 (GNU/Linux)

iD8DBQFMhguY4A2qNGjfk/cRAp+DAKD2cFdN4aHZomU87XhhA2r7GalWcACgt/oM
ffObwWjd44FA6kx5gx/RLDQ=
=DtVE
-----END PGP SIGNATURE-----
\end{verbatim}

Currently, a fast alarm goes directly from \superk to the SNEWS system
with no human intervention.  However, that alarm does not contain
pointing information.  Now, that information is released only after a
virtual meeting of \superk collaborators to confirm the alarm.
However, it is recognized that this step slows down dissemination, so
discussion is starting on the best way to pass this information either
directly to projects like LSST or through systems like SNEWS.

LSST could decide to require multiple or single experimental alarms,
still taking the pointing system from \superk.  In the future, SNEWS
is also expected to supply pointing based on triangulation using
timing and this information could be combined with the electron
scattering signal.

\subsection{High-level description}

We propose a new ``Supernova watch mode'' that would be triggered by a
neutrino supernova alert and would stay engaged for the rest of the
observing day.  Schematically the sequence of events would be as follows:

\begin{itemize}
\item Receive neutrino alert
\item Go into SN watch mode
\item Choose exposure, filter and dither plan
\item Begin a continuous exposure mode in the indicated region of the
  sky attempting to reach limiting depth in the target region.
\item Perform fast DIA analysis to locate CCSN candidate
\item Pass information about the CCSN candidate to community to allow
  other followup observations.
\item Follow the light curve for the rest of the night of observation.
\end{itemize}

\subsection{Footprint -- pointings, regions and/or constraints}

LSST can expect to receive an indicted region of the sky with a target
RA and Dec and (for example) and a half angle opening region with a
68\% coverage.  We advocate covering 99\% of the
localization area with a strategy of first going for maximum coverage
with a single pointing and then covering the rest of the region with
overlap on that central pointing to add depth.  The distance and
luminosity of the CCSN could result in pointing resolution ranging 
from 1 to several degrees on the sky with a typical 10~kpc supernova
being localized to about a 6 degrees FOV.

\subsection{Image quality}

The image quality is not relevant.  Exposures should be taken if at all possible.

\subsection{Individual image depth and/or sky brightness}

Single visits have a depth of 23.14,  24.54, 24.20, 23.65, 22.77,
21.92 in ugrizy.  We advocate attempting to reach magnitude 26 until
the optical counterpart is seen.

\subsection{Co-added image depth and/or total number of visits}

In order to reach magnitude 26 we would need XXXX visits in ugrizy.

(calculate using 2.5*log10(sqrt(visits)) to get to depth you want (e.g. 26th magnitude))

\subsection{Number of visits within a night}

This mode would be envisioned to completely take over the facility for
one dark period after the alarm arrives.

\subsection{Distribution of visits over time}

Edit after discussion:

'I would try to do this at least three times that first night depending
on how long that takes.  Then once the following night, and one more
time a week later just to be sure there was nothing else.' 

\subsection{Filter choice}

We suggest attempting uniform depth in griz in order to make sure we
cover faint and reddened objects.

\subsection{Exposure constraints}

In Figure 7 of~\cite{2016MNRAS.461.3296N} Nakumura and all show that
the extincted optical signal in the visible for a galactic center CCSN
could range from magnitude 5 to 26. The apparent magnitude of the
light curve is a strong function of galactic position (shown in Figure
8) of the same paper.  A dynamic exposure scheme were the exposure is
shortened in positions of little extinction could be considered.

\subsection{Other constraints}

No other relevant constraints.  Clearly the target CCSN would need to be
visible to LSST to enter a supernova watch mode.  Depending on the
time of day, and time delay we might enter watch mode as soon as it
become dark.

\subsection{Estimated time requirement}

We expect no more than one night in the 10 year survey would be
affected by this program. With an expected rate of 3 per century there
is a 22\% chance one CCSN would be seen and a 3\% chance there could
be two.

\subsection{Technical trades}

Not relevant for this proposal.

\section{Performance Evaluation}

Not relevant for this proposal.  Performance would be measured in successfully
identifying and quickly notifying the community of the location of the
of the supernova.

\section{Special Data Processing}

A version of the DIA pipeline would need to be utilized to make the
initial identification.  As one of the advantages of leveraging LSST
for this work is it's ability to see CCSN which have been obscured by
dust in the galactic plane, we also advocate building a set of
templates in the galactic plane to use if the neutrino signal points
us to that region of the sky.

\section{Acknowledgments}

The authors would like to thank Kate Scholberg of Duke University for
information on the SNEWS system and general information, Evan O'Connor
of the University of Stockholm for very useful information and
references related to the modeling of CCSN and properties of the
progenitor stars and light curves, and Super-Kamiokande Spokesperson
Masayuki Nakahata for information on Super-K performance and plans.

\section{References}
% Shared ADS library: https://ui.adsabs.harvard.edu/#/public-libraries/tD3_JEzETqaCHFp83Dk74g

\bibliographystyle{hunsrt} 
\bibliography{references}

\end{document}
